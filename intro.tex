Metaphor is a fundamental cognitive ability that underlies the human capability 
for abstract reasoning. 
Conceptual metaphor theory established the relationship
between metaphors that exist in the mind, \emph{conceptual metaphors}, and 
metaphors used in communication, \emph{linguistic metaphors}. Linguistic metaphors
in a message reveal conceptual metaphors of the sender and
are suggestive of the conceptual metaphors of the receiver, if we assume
sender and receiver share common communicative ground \cite{Clark1996}.
Concrete concepts that are directly grounded in physical experience 
form the basis of abstract concepts \cite{Lakoff1980}. As such, quantifying 
the relationship between culture and context
on metaphor use is an important task \cite{Kovecses2010}.
Here we provide novel time-resolved observations and explanatory
dynamical models of metaphor use on cable news. 
Specifically, 
we recorded metaphorical violence (MV) usage from September to November
of 2012 and 2016 from two shows from three channels, 
MSNBC, CNN, and Fox News. A major contribution presented here is the open-source 
cyberinfrastructure and computational pipeline created for this specific task, 
but with broad applications
to collaborative, time-resolved corpus annotation and analysis of the
human behavior of cable news production. We quantified the details and dynamics of metaphor
use over the three months. This revealed the
degree to which cable news outlets act as promoters, expectation-setters, and
ideological agents, as well as reporters, in different degrees in response to
differing cultural situations in the two study years, 2012 and 2016. 
We anticipate our observations and cyberinfrastructure will be used
as a starting point for many subsequent studies. These studies could include 
developing more realistic
treatments for behavioral experiments, an expanded analysis of 
other revealing types of language use in 
political discourse, and studies of 
the multi-modal (video, audio, text) datasets provided by our 
data source, the Internet Archive's TV News Archive (TVNA).

The production and
comprehension of many classes of linguistic metaphors have been shown to 
recruit other mental pathways. Metaphor processing is not independent of
our motor and emotional circuits \cite{Gallese2005,Lakoff2014,Gibbs2017a}.
In a 2014 study, Kalmoe found that reading metaphorical violence 
resulted in increased support for political violence \cite{Kalmoe2014}. 
This result prompted us to ask just how frequent MV use is on
cable news, still one of the most influential news sources in the internet age. 
In this study, our observations will tell us about differences
in usage between cable news outlets. Eventually, these observations can be
used as treatments for ``natural experiments'' in the 
spirit of \cite{Fusaroli2015}. 
An utterance of metaphorical violence (MV) is one under which the proposition $P$, 
``this event did or could reasonably happen,'' is false.  The utterance
also had to be about politics specifically.
Consider the following examples from 2012, the first metaphor, the second not:
\begin{exe}
  \ex Because we want you to pay for your own birth control, that's an 
    attack on your womb like we're flying a predator drone over your 
    fallopian tubes and calling in a strike?\footnote{Adam 
    Carolla on \emph{The O'Reilly Factor}; \url{https://goo.gl/jVBsqH}} 
    \label{ex:carolla}
  \ex John McCain and his allies have been trying to turn the Benghazi 
    attacks into a political scandal for the president since 
    September.\footnote{Chris Matthews on his show, \emph{Hardball with Chris Matthews}; 
    \url{https://goo.gl/Pfs4Sc}} \label{ex:benghazi-attacks}
\end{exe}
In \ref{ex:carolla}, Adam Corolla is criticizing Sandra Fluke for her efforts to 
force all employers to provide insurance covering birth control for women. 
Clearly $P$ is false for the event of ``an attack'' on anyone's ``womb'' by
means of ``a predator drone,'' meaning \ref{ex:carolla} is a metaphor.
Metaphor is ultimately a mapping between concepts. 
Elements from the \emph{source domain} concept are mapped on to the \emph{target domain} 
concept. These concepts can be specified to varying degrees of specificity as
the application may require. In our case it suffices to identify \emph{violence}
as the source domain and \emph{politics} as the target domain. Said a different
way, metaphorical violence is an expression of understanding politics in
terms of violence. We call the word that signals, or instantiates, the source
domain of violence the \emph{violence signal}. We consider only phrases that
use one of three violence signals: \emph{attack}, \emph{beat}, or \emph{hit}. 
These three were the most common among a list of about twenty 
violent words we initially considered. There are many other possible violence 
signals that we leave as future work to investigate. 

\subsection{Motivation}

Metaphor in political communication is hypothesized to heighten \emph{pathos},
the feeling part of reason that Aristotle proposed had a major 
influence on how we think \cite{Charteris-Black2009}.  Returning to Example \ref{ex:carolla}
some may find Carolla's utterance offensive or humorous, 
but in either case Carolla achieves a heightened emotional experience 
through metaphorical violence.
Since the dawn of conceptual metaphor theory,
marked by the publishing of \emph{Metaphors We Live By}~\cite{Lakoff1980},
metaphor researchers have pushed to find a physical theory of metaphor in the
brain \cite{Feldman2006}. In order to guide such a search, 
it's necessary first to understand correlations between metaphor and behavior more broadly. 
Consider a common non-violent example: someone says ``my career has come to a fork.'' 
Conceptual metaphor theory researchers have identified that 
comprehension of such an utterance relies on
co-activation of brain regions associated with everyday naviagation as well
as regions associated with more abstract decision-making.
Brain regions associated with 
navigation but not with ``careers'' are said to be \emph{recruited} for
simulating the experience of navigating~\cite{Croft2005,Matlock2004,Gibbs2008}.
Just how MV recruits non-language-specific brain regions is an open question.

The first, now famous, televised presidential debate was between then-vice president 
Richard Nixon, the Republican party candidate, and then-senator John F. Kennedy,
the Democratic party candidate. In promoting the first TV debate, TV networks avoided
using the word \emph{debate} for fear of framing one candidate as ``winning'' 
and the other ``losing''---the important civic task at hand 
was supposed to be an elevated one. 
It was a chance to choose the best course for the nation.
However, after this first TV debate the 
press and public mostly agreed that JFK indeed won. 
Audiences were fascinated by this new political theater
in their living rooms, so the media shed many of its 
reservations for sports and violence
metaphors for describing what was done or should be done in debates 
\cite[p. 99]{Schroeder2008}. This trend toward increased spectacle and 
competitive framing has continued since then. On the morning of the first
1992 presidential debate, on the ABC show \emph{This Week with David Brinkley},
columnist George Will compared the debates to the Super Bowl, 
saying ``Each year the hyperbole and rubbish and pageantry
and marching bands surrounding the little kernel of football in the middle gets
larger and larger'' \cite[p. 111]{Schroeder2008}. 
In 2016, \emph{Monday Night Football} ratings for September 26
were at a historic low, its usual viewership eaten up by the Clinton-Trump 
debate on the same night.
That debate was the most-watched presidential debate in history, with an estimated 84
million viewers. Of those 84 million, 26.2 million watched on
MSNBC, CNN, or Fox News \cite{Perlberg2016}.  For comparison, the 2016 Super Bowl, the 
most-watched TV event in history, raked in 111.8 
million viewers~\cite{Hagemann2016}.

The influence and ideological leanings of cable news make it an important 
system to study and understand.  The three channels we studied were 
MSNBC, CNN, and Fox News, the most-watched cable news channels \cite{OConnell2017}. 
In a 2014 study, survey respondents most frequently mentioned CNN and 
Fox News as the source they ``turn to most often for news about government and 
politics.'' As part of the same study, an ``ideological profile'' was created
for each of these three networks, along with other news outlets, on TV, 
in print, and on the web. 
CNN and MSNBC were close to each other on the
liberal side of this scale, while Fox News was nearly exactly oppositely 
placed on the conservative side of this ideology scale~\cite{Pew2014}.
During the lead-up to the 2016 presidential election, 40\% of Trump voters
said Fox News was their primary source of news, and MSNBC and CNN combined
accounted for 27\% of Clinton voters primary source of news~\cite{Pew2017}.
We observed MV usage on three cable news channels from 
September 1 to November 30 in study years 2012 and 2016. 
During these dates, four major political TV events happen: the 
three presidential elections, all but one occurring in October, 
and the presidential election, called ``Voting Day,'' 
which is always the first Tuesday in November after November 1.

In addition to revealing details of MV use on US cable news for the study
of political communication and action, our study
provides more data for understanding a deep question in 
cognitive linguistics: to what extent does the cultural context
influence which metaphors are used~\cite{Gibbs1997,Kovecses2010}?
The major results of our study are a series of observations about MV use
across different cultural situations and by different cultural actors. 
Metaphor use is one of many distinguishing markers of a culture and its subcultures. 
We want to support two intentionally vague hypotheses about
culture and metaphor for our specific system under investigation. 
First, we postulate that MV use changes 
in response to, or in anticipation of, the cultural events of the 
presidential debates and Voting Day.
We also expect metaphor use to differ between the three channels. 
However, we also expect some similarities, due to the shared cultural 
frames between the three networks. We left these \emph{a priori} hypotheses
vague because any hypothesis would simply be a guess. We opt instead to take
an observational approach, to let
the data speak for itself. We structure our dynamical observations
with the aid of a simple model to 
act as an information processing system \cite{McElreath2016}.
