\subsection{Additional figures and tables}

%%
% Table with details of the model fits
%
\begin{table}[!h]
  \centering

  \begin{subtable}{\linewidth}
    \centering
    \begin{tabular}{lllrrrr}
\toprule
{} & $t_0^{(2)}$ & $t^{(2)}_{N^{(2)}}$ & $f^{(1)}$ & $f^{(2)}$ & \emph{delta} & total uses \\
\midrule
MSNBC    &  2012-09-13 &          2012-09-27 &      2.03 &      3.19 &       0.57 &        283 \\
CNN      &  2012-11-07 &          2012-11-30 &      2.02 &      0.58 &      -0.71 &        213 \\
Fox News &  2012-09-09 &          2012-09-25 &      2.41 &      1.65 &      -0.31 &        262 \\
\bottomrule
\end{tabular}

    \caption{\quad2012; total uses: 758}
    \label{tab:fit-parameters-2012}
  \end{subtable}
  
  \vspace{.25in}

  \begin{subtable}{\linewidth}
    \centering
    \begin{tabular}{lllrrrr}
\toprule
  {} & $t_0^{(2)}$ & $t^{(2)}_{N^{(2)}}$ & $f^{(1)}$ & $f^{(2)}$ & \emph{delta} & total uses \\
\midrule
MSNBC    &  2016-10-08 &          2016-10-26 &      0.85 &      2.04 &       1.40 &        126 \\
CNN      &  2016-09-23 &          2016-10-27 &      1.14 &      2.81 &       1.45 &        196 \\
Fox News &  2016-09-23 &          2016-10-22 &      1.12 &      3.60 &       2.22 &        261 \\
\bottomrule
\end{tabular}

    \caption{\quad2016; total uses: 583}
    \label{tab:fit-parameters-2016}
  \end{subtable}

  \caption{Parameters for the best-fit (most likely) model for 2012 and 2016.}
  \label{tab:fit-parameters}
\end{table}

%%
% Table with network and violent words
%
\begin{table}[H]
  \centering

  \begin{subtable}{\linewidth}
    \centering
    \begin{tabular}{lclrrrr}
\toprule
       &  &   & $f^{(1)}$ & $f^{(2)}$ & reactivity & total uses \\
Violent Word & total uses & Network &           &            &            \\
\midrule
hit & 142 & MSNBC &      0.86 &      0.86 &      -0.00 &         67 \\
     && CNN &      0.54 &      0.11 &      -0.81 &         34 \\
     && Fox News &      0.57 &      0.33 &      -0.42 &         41 \\
     \hline
beat & 200 & MSNBC &      1.03 &      1.64 &       0.59 &         89 \\
     && CNN &      0.66 &      0.63 &      -0.04 &         51 \\
     && Fox News &      0.83 &      0.53 &      -0.35 &         60 \\
     \hline
attack & 416 & MSNBC &      1.30 &      3.14 &       1.42 &        127 \\
     && CNN &      2.07 &      0.32 &      -0.85 &        128 \\
     && Fox News &      2.08 &      2.00 &      -0.04 &        161 \\
\bottomrule
\end{tabular}

    \caption{\quad 2012}
    \label{tab:words-2012}
  \end{subtable}
  
  \vspace{.25in}

  \begin{subtable}{\linewidth}
    \centering
    \begin{tabular}{llrrrr}
\toprule
       &     & $f^{(1)}$ & $f^{(2)}$ & \emph{delta} & total uses \\
Violent Word & Network &           &           &            &            \\
\midrule
hit & MSNBC &      0.16 &      0.06 &      -0.63 &         10 \\
       & CNN &      0.27 &      0.45 &       0.64 &         25 \\
       & Fox News &      0.46 &      1.36 &       1.97 &         56 \\
\hline
beat & MSNBC &      0.54 &      1.47 &       1.75 &         55 \\
       & CNN &      0.50 &      0.79 &       0.59 &         45 \\
       & Fox News &      0.48 &      0.88 &       0.84 &         45 \\
\hline
attack & MSNBC &      0.61 &      1.59 &       1.62 &         61 \\
       & CNN &      1.16 &      2.59 &       1.23 &        126 \\
       & Fox News &      1.08 &      4.32 &       2.99 &        160 \\
\bottomrule
\end{tabular}

    \caption{\quad 2016}
    \label{tab:words-2016}
  \end{subtable}

  \caption{Uses and \emph{delta} for violence signals on each network in 2012 and 2016.}
  \label{tab:words}
\end{table}

% \begin{table}[!h]

%   \centering
%   \bgroup
%     \begin{subtable}{\textwidth}
%       \centering
%       \begin{tabular}{lr}
%         \toprule
%         Program name & Total MV uses \\
%         \midrule
%         The Rachel Maddow Show (MSNBC) & 93 \\
%         Hardball With Chris Matthews (MSNBC) & 208 \\
%         Anderson Cooper 360 (CNN) & 99 \\
%         Piers Morgan Tonight (CNN) & 118 \\
%         The O'Reilly Factor (Fox News) & 141 \\
%         Hannity (Fox News) & 133 \\
%         \bottomrule
%       \end{tabular}
%       \caption{Total uses by show in 2012}
%       \label{tab:by-show-2012}
%     \end{subtable} \\  \vspace{1.5em}
%     \begin{subtable}{\textwidth}
%       \centering
%       \begin{tabular}{lr}
%         \toprule
%         Program name & Total MV uses \\
%         \midrule
%         The Rachel Maddow Show (MSNBC) & 66 \\
%         The Last Word with Lawrence O'Donnel (MSNBC) & 80 \\
%         Anderson Cooper 360 (CNN) & 100 \\
%         Erin Burnett OutFront (CNN) & 118 \\
%         The O'Reilly Factor (Fox News) & 146 \\
%         The Kelly File (Fox News) & 148 \\
%         \bottomrule
%       \end{tabular} \quad
%       \caption{Total uses by show in 2016}
%       \label{tab:by-show-2016}
%     \end{subtable}

%   \egroup
%   \caption{Total uses by show in each of the two years}
%   \label{tab:by-show}
% \end{table}

% \clearpage
% \begin{figure}[!h]
%   \centering
%     \begin{subfigure}{0.9\linewidth}
%       \centering
%       \includegraphics[width=\textwidth]{Figures/2012-all.pdf}
%       \caption{2012}
%       \label{fig:2012-all-reg}
%     \end{subfigure} \\[2em]
%     \begin{subfigure}{0.9\linewidth}
%       \centering
%       \includegraphics[width=\textwidth]{Figures/2016-all.pdf}
%       \caption{2016}
%       \label{fig:2012-all-reg}
%     \end{subfigure}
%   \caption{Metaphorical violence tracked candidate tweets significantly in 2012 and
%     2016. Frequency is over all networks and subject/object relationships.
%     Correlation coefficient is greater in 2016 than 2012, supporting other 
%     claims of 2016 being a ``Twitter election.''
%   }
%   \label{fig:regressions-all}
% \end{figure}

% \clearpage

% \begin{figure}[!h]
%   \centering
%     \includegraphics[width=0.95\textwidth]{Figures/2012-network.pdf}
%   \caption{Regressions of metaphorical violence on each network in response
%    to tweets from @BarackObama and @MittRomney.}
%   \label{fig:2012-network}
% \end{figure}


% \begin{figure}[H]
%   \centering
%     \includegraphics[width=0.95\textwidth]{Figures/2016-network.pdf}
%   \caption{Regressions of metaphorical violence on each network in response
%    to tweets from @HillaryClinton and @realDonaldTrump.}
%   \label{fig:2016-network}
% \end{figure}


% \begin{figure}[H]
%   \centering
%     \includegraphics[width=0.9\textwidth]{Figures/2012-subjobj.pdf}
%   \caption{Regressions of metaphorical violence faceted by subject and object
%     in response to to tweets from @BarackObama and @MittRomney.}
%   \label{fig:2012-subjobj}
% \end{figure}


% \begin{figure}[H]
%   \centering
%     \includegraphics[width=0.9\textwidth]{Figures/2016-subjobj.pdf}
%   \caption{Regressions of metaphorical violence faceted by subject and object
%     in response to to tweets from @HillaryClinton and @realDonaldTrump.}
%   \label{fig:2016-subjobj}
% \end{figure}

% \clearpage

% \subsection{Notes on model fitting}

% We presented both Bayesian-type relative likelihood inference probabilities
% for the fitted models as well as the frequntist p-value measure for each 
% candidate model parameterization. Frequentists can breathe a sigh of relief, 
% since all but one model's optimal paramaterization had a p-value greater than
% 0.01; it was 0.011. The other five optimal models had p-value significance 
% less than 0.01 (Tables~\ref{tab:relative-likelihoods-2012} 
% and~\ref{tab:relative-likelihoods-2016}).  While the significance can
% tell us whether or not a model is appropriate for the data, there were many 
% models that were significant from a frequentist perspective. The relative
% likelihood of alternative parameterizations tells us the probability that an
% alternative parameterization will capture more information about the dynamics
% than the parameterization with minimal AIC, our chosen information metric.
% In an inexact sense, the various relative likelihoods tell us the weight of
% each parameterization in some superordinate representation that is a
% weighted sum of all potential parameterizations. Exactly this is done in some
% applications, where a weighted sum of models is selected as the optimal model, 
% typically models with a differing number of independent dimensions.
% The presence of either greater or fewer large relative likelihoods signals either
% a stronger or weaker presence of the impulse-type behavior we have hypothesized.
% In 2012, the presence of an impulse was much weaker, with the next nine most
% likely parameterizations having a relative likelihood greater than 0.5, with
% two exceptions of .47 and .41 for MSNBC (Table~\ref{tab:supp-msnbc-2012}). 
% In 2016, only one alternative had a relative
% likelihood greater than 0.5, also on MSNBC: 0.508 
% (Table~\ref{tab:supp-msnbc-2016}). So in this sense, MSNBC had the best-defined
% impulse in 2012 and the worst-defined impulse in 2016.

% In 2012, two networks showed negative reactivity.
% In one case, CNN's MV usage dips after election
% day, perhaps there was no longer a cultural driving force for MV usage
% because there was no more ``fighting'' in the debates or elections. 
% The other case where
% reactivity was negative was Fox News' MV usage in 2012. This may be because
% Obama was perceived as the more skilled debater and the Republican side 
% avoided raising emotions and expectations over the debates, or other reasons
% discussed later. MSNBC showed positive reactivity in 2012, 
% more than doubling its frequency from its initial base level to its 
% modulated level.  In 2016, the three networks showed greater coherence in their 
% behavior.  Fox News was the most reactive, followed closely by CNN. The time period
% of Fox News' modulated state lasted ten days longer than CNN's. MSNBC was 
% the least reactive, but with a significant reactivity of 163\% 
% (Table~\ref{tab:fit-parameters}).




% \begin{table}[!h]
%   \centering

%   \bgroup
%     \begin{subtable}{\linewidth}
%       % \hfill
%       \centering
%       \begin{tabular}{rllrr}
\toprule
 rel. lik. & $t_0^{(2)}$ & $t^{(2)}_{N^{(2)}}$ & reactivity &  $P(<|t|)$ \\
\midrule
  1.000000 &  2012-09-12 &          2012-09-26 &       0.94 &   0.000326 \\
  0.951229 &  2012-09-09 &          2012-09-26 &       0.90 &   0.000344 \\
  0.582086 &  2012-09-09 &          2012-09-27 &       0.85 &   0.000572 \\
  0.565037 &  2012-09-12 &          2012-09-27 &       0.88 &   0.000590 \\
  0.506725 &  2012-09-10 &          2012-10-17 &       0.79 &   0.000660 \\
  0.431549 &  2012-09-10 &          2012-10-18 &       0.79 &   0.000780 \\
  0.397452 &  2012-09-09 &          2012-10-04 &       0.77 &   0.000850 \\
  0.379520 &  2012-09-10 &          2012-09-28 &       0.81 &   0.000892 \\
  0.354267 &  2012-09-10 &          2012-10-16 &       0.76 &   0.000959 \\
  0.346991 &  2012-09-12 &          2012-09-28 &       0.83 &   0.000980 \\
\bottomrule
\end{tabular}

%       \caption{MSNBC; 2012}
%       \label{tab:supp-msnbc-2012}
%     \end{subtable} \\  \vspace{1.5em}

%     \begin{subtable}{\linewidth}
%       \centering
%       \begin{tabular}{rllrr}
\toprule
 rel. lik. & $t_0^{(2)}$ & $t^{(2)}_{N^{(2)}}$ & reactivity &  $P(<|t|)$ \\
\midrule
  1.000000 &  2012-11-06 &          2012-11-29 &      -0.70 &   0.000148 \\
  0.875496 &  2012-10-26 &          2012-11-29 &      -0.61 &   0.000170 \\
  0.772116 &  2012-11-07 &          2012-11-29 &      -0.71 &   0.000193 \\
  0.607375 &  2012-11-08 &          2012-11-30 &      -0.72 &   0.000247 \\
  0.514332 &  2012-11-06 &          2012-11-28 &      -0.70 &   0.000293 \\
  0.427936 &  2012-10-30 &          2012-11-30 &      -0.60 &   0.000354 \\
  0.405132 &  2012-11-07 &          2012-11-28 &      -0.70 &   0.000375 \\
  0.375951 &  2012-10-02 &          2012-10-15 &       1.18 &   0.000405 \\
  0.352062 &  2012-09-27 &          2012-10-15 &       1.08 &   0.000433 \\
  0.324736 &  2012-11-08 &          2012-11-28 &      -0.71 &   0.000471 \\
\bottomrule
\end{tabular}
  
%       \caption{CNN; 2012}
%         \label{tab:supp-cnn-2012}
%     \end{subtable} \\  \vspace{1.5em}

%     \begin{subtable}{\linewidth}
%       \centering
%       \begin{tabular}{rllrr}
\toprule
 rel. lik. & $t_0^{(2)}$ & $t^{(2)}_{N^{(2)}}$ & reactivity &  $P(<|t|)$ \\
\midrule
  1.000000 &  2012-09-12 &          2012-09-24 &      -0.55 &   0.025549 \\
  0.917421 &  2012-09-11 &          2012-09-24 &      -0.52 &   0.028131 \\
  0.860348 &  2012-09-09 &          2012-09-24 &      -0.50 &   0.030230 \\
  0.820812 &  2012-09-07 &          2012-09-24 &      -0.48 &   0.031871 \\
  0.646732 &  2012-09-11 &          2012-09-23 &      -0.51 &   0.041741 \\
  0.614601 &  2012-09-07 &          2012-11-20 &      -0.36 &   0.044240 \\
  0.607754 &  2012-09-10 &          2012-09-23 &      -0.48 &   0.044810 \\
  0.583962 &  2012-09-08 &          2012-11-26 &      -0.42 &   0.046906 \\
  0.581079 &  2012-09-07 &          2012-09-22 &      -0.46 &   0.047173 \\
  0.507553 &  2012-09-25 &          2012-10-06 &       0.54 &   0.055126 \\
\bottomrule
\end{tabular}
 
%       \caption{Fox News; 2012}
%       \label{tab:supp-fox-2012}
%     \end{subtable}
%   \egroup

%   \caption{Relative likelihood of the null model and the ten most-likely
%     dynamical models with alternative parameterizations. Parameterizations given
%     are the first and last date of the excited state.}
%   \label{tab:relative-likelihoods-2012}
% \end{table}

% \begin{table}[!h]
%   \centering
%   \bgroup
%     \begin{subtable}{\linewidth}
%       % \hfill
%       \centering
%       \begin{tabular}{rllrr}
\toprule
 rel. lik. & $t_0^{(2)}$ & $t^{(2)}_{N^{(2)}}$ & reactivity &  $P(<|t|)$ \\
\midrule
  1.000000 &  2016-10-08 &          2016-10-26 &       1.40 &   0.000015 \\
  0.385238 &  2016-10-08 &          2016-10-25 &       1.34 &   0.000039 \\
  0.338263 &  2016-10-08 &          2016-11-03 &       1.28 &   0.000044 \\
  0.220983 &  2016-10-06 &          2016-10-26 &       1.26 &   0.000068 \\
  0.158223 &  2016-10-11 &          2016-10-26 &       1.29 &   0.000096 \\
  0.158223 &  2016-10-10 &          2016-10-25 &       1.29 &   0.000096 \\
  0.158002 &  2016-10-08 &          2016-11-05 &       1.21 &   0.000096 \\
  0.126735 &  2016-10-09 &          2016-11-03 &       1.19 &   0.000120 \\
  0.115298 &  2016-10-07 &          2016-11-03 &       1.18 &   0.000132 \\
  0.114875 &  2016-10-08 &          2016-11-06 &       1.19 &   0.000133 \\
\bottomrule
\end{tabular}

%       \caption{MSNBC; 2016}
%       \label{tab:supp-msnbc-2016}
%     \end{subtable} \\  \vspace{1.5em}

%     \begin{subtable}{\linewidth}
%       \centering
%       \begin{tabular}{rllrr}
\toprule
 rel. lik. & $t_0^{(2)}$ & $t^{(2)}_{N^{(2)}}$ & reactivity &  $P(<|t|)$ \\
\midrule
  1.000000 &  2016-09-25 &          2016-10-27 &       1.45 &   0.000073 \\
  0.764196 &  2016-09-24 &          2016-11-04 &       1.55 &   0.000095 \\
  0.760063 &  2016-09-25 &          2016-10-13 &       1.46 &   0.000096 \\
  0.661919 &  2016-09-25 &          2016-11-03 &       1.50 &   0.000110 \\
  0.654467 &  2016-09-27 &          2016-10-27 &       1.40 &   0.000112 \\
  0.648921 &  2016-09-26 &          2016-10-13 &       1.46 &   0.000113 \\
  0.537424 &  2016-09-23 &          2016-10-20 &       1.37 &   0.000136 \\
  0.450599 &  2016-09-26 &          2016-11-04 &       1.45 &   0.000163 \\
  0.399820 &  2016-09-26 &          2016-11-03 &       1.41 &   0.000184 \\
  0.398737 &  2016-09-23 &          2016-10-12 &       1.41 &   0.000185 \\
\bottomrule
\end{tabular}
  
%       \caption{CNN; 2016}
%         \label{tab:supp-cnn-2016}
%     \end{subtable} \\  \vspace{1.5em}

%     \begin{subtable}{\linewidth}
%       \centering
%       \begin{tabular}{rllrr}
\toprule
 rel. lik. & $t_0^{(2)}$ & $t^{(2)}_{N^{(2)}}$ & reactivity &     $P(<|t|)$ \\
\midrule
  1.000000 &  2016-09-24 &          2016-10-22 &       2.22 &  1.991356e-11 \\
  0.270536 &  2016-09-23 &          2016-10-24 &       2.17 &  7.238780e-11 \\
  0.041894 &  2016-09-25 &          2016-10-25 &       2.08 &  4.577034e-10 \\
  0.031166 &  2016-09-22 &          2016-10-23 &       2.03 &  6.134216e-10 \\
  0.019854 &  2016-09-23 &          2016-10-21 &       1.96 &  9.586853e-10 \\
  0.015325 &  2016-09-23 &          2016-10-26 &       2.05 &  1.239016e-09 \\
  0.012477 &  2016-09-16 &          2016-10-23 &       2.20 &  1.518961e-09 \\
  0.011004 &  2016-09-21 &          2016-10-23 &       1.99 &  1.720502e-09 \\
  0.006018 &  2016-09-16 &          2016-10-24 &       2.19 &  3.130088e-09 \\
  0.004315 &  2016-09-21 &          2016-10-24 &       1.95 &  4.354356e-09 \\
\bottomrule
\end{tabular}
 
%       \caption{Fox News; 2016}
%       \label{tab:supp-fox-2016}
%     \end{subtable}
%   \egroup

%   \caption{Relative likelihood of the null model and the ten most-likely
%     dynamical models with alternative parameterizations. Parameterizations given
%     are the first and last date of the excited state.}
%   \label{tab:relative-likelihoods-2016}
% \end{table}

