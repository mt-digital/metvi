Consilience is the general trend towards integration of seemingly disparate
ways of knowing. Cognitive science is a consilient science: there is a recognition
of the unity of at least six branches of science: linguistics, computer science,
psychology, philosophy, neuroscience, and anthropology. Anthropology and 
neuroscience anchor cognitive science as a study of humans. The study of other
animals can be beneficial to help us understand human behavior:
cognitive science is a biological science. Because the mechanical function of 
the brain is so complex, we lack a complete understanding of its function. 
Philosophy and psychology have overcome this limitation for millenia to 
postulate and later discover scientific truths about the mind. One way we can
observe human behavior is by analyzing human communication, which takes on many
dimensions. All communication involves transforming conceptual relationships
somehow stored and activated in the brain to bodily mechanical action, including
such actions as producing pheromones. Linguistics is the scientific study of
communication. Cognitive linguistics seeks ecologically consistent explanations
of communication: this will include not just conceptual relationships, but also
the emotional and cultural state of the communicator-environment 
system. This paper imagines what a true unification of these fields might look
like by suggesting a foundational formal and conceptual model based on 
energy- and information-driven evolution. I use that principle to derive 
a number of models from each of the subdisciplines of cognitive science,
except for philosophy. Philosophical results will guide our inquiry, including
the task of defining contested concepts between the fields we are trying to
consolidate.
