We collected cable news transcripts indexed by date, cable channel,
and program name. We identified and counted daily MV usage that used
violence signals \emph{attack}, \emph{hit}, and \emph{beat}.
In a phrase where violence is done, the person doing the voice and the person
to whom the violence is done are called the subject and object of MV. 
We counted the daily instances where either the Democrat or Republican 
presidential candidate appeared as the subject or object of MV. An open 
question is, how do metaphor and culture influence one another dynamically?
To assess the dynamics of this system of cable news-produced MV with 
culture at large, we fit a model for six timeseries, one for each 
channel in each study year.  We calculated reactivity across networks, 
violence signals, and subject and object cross-sections.

We begin by reporting total MV use over the three-month study periods 
in 2012 and 2016. We observed 758 uses in 2012 and 583 uses in 2016. 
In 2012, \emph{Hardball} alone contained 208 metaphorical violence uses.
MSNBC shows other than \emph{Hardball} ranged from 60 to 120 across the two years. 
The CNN shows' MV usages were 99, 100, 118, and 118, showing a more consistent
level of usage than MSNBC.  Fox News shows were similarly constrained to a 
narrower count range.  Across both years we observed between 130 and 150 MV uses. 
See Table~\ref{tab:by-show} in the Supplement for full tables.

Next we report usage for each violence signal by each channel. 
The general distribution of usage was similar in both years:
\emph{attack} was the most-used, followed by \emph{beat} as next most-used in
both years, and \emph{hit} was the least-used in both study years. Interestingly,
in 2012 when MSNBC led in total MV use, MSNBC used 
violence signals \emph{hit} and \emph{beat} more often than
\emph{attack}. Another consistency between study years was that the Republican 
candidate in both years was both the subject and object of 
metaphorical violence more often than their Democrat counterparts. Exact
quantities and more detailed data are given in Tables~\ref{tab:fit-parameters},~\ref{tab:words}, 
and~\ref{tab:subjobj}.

Next, we calculated and compared the reactivity of MV. 
Bayesian multi-model inference allowed us to identify the best-fit model and
quantify the relative likelihood of other parameterizations. 
The top-ten alternative models by relative
likelihood are given in the supplement. To further support 
the meaningfulness of the these best-fitting parameterizations, 
we calculated p-values. All fits were
significant, with the greatest p-value being 0.026 (Fox News in 2012), 
and all others less than 0.001 (see Tables~\ref{tab:relative-likelihoods-2012} 
and~\ref{tab:relative-likelihoods-2016} for full details).

We found instances of positive and negative reactivity values, 
meaning that MV did not uniformly increase within the
three study months for all channels. 
Fox News and CNN had negative reactivities in 2012. 
In the case of Fox News in 2012, 
there was a decrease in MV frequency starting September 9 and ending 
September 25, the days leading up to the first presidential debate on October 3.
CNN's MV usage dipped after November 6, Voting Day, starting 
November 7 and continuing to the last study date, November 30. 
In 2012, MSNBC was the only channel with
a positive reactivity, starting on September 13 and ending September 27,
before the first debate. In 2016, the reactivity was positive and larger
in magnitude for all three channels, and the dates of State 2 
overlapped to a much greater degree.  

MSNBC's positive reactivity was due 
mainly to an increased use of constructions using \emph{attack}. There was
zero reactivity with the \emph{hit} violence signal on MSNBC. For CNN, their use
of \emph{hit} and \emph{attack} decreased the most, by about 80\%.
On Fox News in 2012, most of the dip in MV use was due to decreases in 
\emph{hit} and \emph{beat}, with uses of MV with \emph{attack} staying nearly
constant. In 2016, reactivities were positive for all networks. All reactivities
for violence signals were also positive, with one exception: MSNBC's use of
\emph{hit} fell by 63\%. MSNBC's use of \emph{beat} and \emph{hit} increased
by a factor of nearly 2. In 2016, CNN's use of \emph{attack} accounted for 
most of its overall increase in MV frequency. 
The same was true for Fox News: their usage of
\emph{attack} increased by nearly 300\%. See Table \ref{tab:words}
for full details of reactivities across violence signals.

In 2012, reactivities for both candidates as
subject and object were relatively small on MSNBC and Fox News, compared to
2016. Since
the election had passed, CNN's use of either candidate as the subject or
object of metaphorical violence decreased to zero or nearly to zero in State 2. 
In 2016 this changed. For Hillary Clinton as subject or object, the
reactivity of Fox News was the largest, increasing its framing of Clinton
as subject by 220\% and as object by 248\%.
Fox News also reacted most strongly in casting 
Donald Trump as the subject or object of metaphorical violence. Fox News cast Trump 
as subject of MV by 284\% more frequently in State 2 and as
object 106\% more frequently in State 2.  
All three networks increased the 
frequency of Donald Trump as the subject and object of metaphorical violence. 
The only negative reactivity among networks with candidates as subject or object 
was MSNBC, which decreased its use of Hillary Clinton as the subject of MV 
in State 2.  See Table \ref{tab:subjobj} for 
full details of reactivites across subject and object specifications.
