Our study revealed differences in the production of metaphorical
violence on cable news between news channels and between two 
election years. There were differences between how much MV was 
produced by each cable news channel in each of the two study years. We found
differences in the timing and magnitude of broad changes in the levels of
MV use over time. There were commonalities, too: \emph{attack} was the most-used 
violence signal. The republican candidates were more often included
in MV as either the subject or the object of MV than the democratic candidates.
Our tabulations were sensitive to which
shows were included. 
Fox News shows consistenly feature the most metaphorical violence if MSNBC's
\emph{Hardball with Chris Matthews} is left out. 
In 2012, MV was used on \emph{Hardball} 40\% more often than Fox News's \emph{The Kelly
File}, which most frequently had uses of MV among Fox News shows (Table~\ref{tab:by-show}).
Republicans were more involved in MV than the democratic candidates. 
In 2016, Donald Trump was the one doing
metaphorical violence 102 times on Fox News compared with Hillary Clinton as
perpetrator of MV only 30 times. 
This could inform with other research that says republicans or southerners 
more often conceptualize interpersonal relationships in terms of violence, or are
more likely to resort to violence in their interpersonal relationships 
\cite{Lakoff1996, Cohen1996}.

In 2012, the election was between Barack Obama, the incumbent, and Mitt Romney.
In many ways, 2012 showed less MV usage and reaction to debates and
the elections than in 2016.  
In 2012, the candidates themselves were involved in less of the metaphorical
violence than in 2016. Two of the three channels in 2012 showed a decrease in
overall metaphorical violence at some point in the three-month study period.
Even the increase on MSNBC was not as pronounced in 2012 as in 2016. In 2012
the reactivity of MSNBC was 0.57 compared with 1.40 in 2016. Not only were the
reactivities uniformly positive and larger in magnitude in 2016, State 2 of
all timeseries began and ended more closely together. We
could say that the different channels reacted more coherently in 2016.
CNN and Fox News showed increased frequency on the same day, September 9, 2016,
with the last day of State 2 coming just five days apart: October 27 for CNN
and October 22 for Fox News.  MSNBC's frequency became elevated later, 
on October 8, but also ended its State 2 around the same time as the
other two channels, on October 26. What could be the cause of this? 
Fox News may have been more subdued in its coverage because of less excitement
about Mitt Romney as a candidate. Perhaps Fox News did not want to attract
attention to the debates due to an expectation that Romney would do poorly
against Barack Obama. In 2016 we had the first woman candidate for president
running against a TV celebrity. That there was less excitement
about the 2012 debates is reflected in the ratings: 67.2 million watched 
the first Obama-Romney debate compared with 84 million viewers of the first
Clinton-Trump debate.

Our results were enabled by thinking deeply about what is required to efficiently
annotate the TVNA for studying metaphor. This consideration led us to develop
a custom data model and software to accommodate the TVNA data and metadata,
providing metadata either as data inputs for analysis or as a tool to 
the annotater, combining these with annotations. The raw and annotated data
is all available online with an accompanying Jupyter Notebook so other 
researchers can critique our annotations and reproduce our results. 
Metaphor identification and annotation is time-intensive. Our system has
enabled a large-scale, time-resolved study of metaphor.
Still, the data we present is admittedly limited. Results are 
sensitive to which shows we study, as illustrated by 
our inclusion of \emph{Hardball} in 2012. Our software system for metaphor annotation
is flexible, allowing for new features to be added to further speed metaphor
identification and annotation. Basic improvements will be to the cooperative
workflow and user interface. An interesting planned extension of Metacorps
will be to add a deep learning-based machine assistant that can learn what 
metaphors look like for each project as users add more 
annotations\footnote{\url{https://github.com/mt-digital/metacorps/issues/15}}\cite{DoDinh2016,Rei2017}.  
Each project's trained deep neural network can be used to suggest annotations
for new projects through transfer learning~\cite[p. 526]{Goodfellow2016a}.

More efficient data collection and annotation enables new experiments
to more deeply understand the dynamical relationship of metaphorical violence on 
attitudes. King, et al., worked closely with online news agencies to
create camoflauged psychological treatments by selecting which news topics
would be published when~\cite{King2017}. They found that discussion of 
the chosen topics on social media correlated with the treated publishing of these 
news stories. While these authors
should be applauded for their persistence in convincing news agencies
to participate in this ethically-delicate and disruptive 
experiment, which took years to nurture and implement, we suggest a simpler
experiment. In order to understand the impact of metaphorical violence,
we can simply record metaphorical violence on TV News, poll people nationwide
to learn their recent TV viewing history, trait-aggressiveness,
and support for real-life political violence as in~\cite{Kalmoe2014}. Then we 
may learn more about people's reaction to MV on TV news. 

Our work has given us a new perspective on metaphorical violence in political
discourse. Before this, we knew metaphorical violence was a feature of 
political communication, but we did not know the extent of it. We have given 
a birds-eye view of the magnitude and dynamics of metaphorical violence
usage on one medium, cable news.
MV use can change dramatically over the course of three months. 
Our results 
underscore the need for a deeper understanding of the impact of MV on 
political decision-making. Clearly sometimes we want to fire people up for
the greater good.
Human rights leaders may want their co-workers to feel brave
and energized, and metaphorical violence might steel them for
humanitarian work in the face of real violence.
However, it may also be the case that metaphorical violence may overly
arouse emotions that distract us, or worse. The
issues brought forth at election time do not go away afterwards;
democracy depends on factions who were ``attacking'' each other to 
come together afterwards to govern effectively. At the extreme end,
some in the population may become conditioned to see politicians or 
citizens from the opposite party as a truly violent threat, and take
violent steps against political opponents. We are certainly not in a place to
establish causation between metaphorical violence on TV News and violent
acts like the June, 2017 shooting of conservative congressmen playing baseball
by a deranged liberal. However, it is not impossible that violent
rhetoric, on TV and elsewhere, could be a confounding
factor in political violence. In a world of ever-more optimized
technology, we should consider how to optimize our political discourse
to get the best possible results. Our results stress that metaphorical 
violence should be a primary discourse feature to evaluate.
