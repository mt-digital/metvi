
\subsection{Data collection and annotation}

Data were collected from the Internet Archive's TV News Archive
(TVNA).\footnote{See \url{https://archive.org/details/tv}.} Using custom
software to access, annotate, and analyze TVNA data, we could effectively
download, review, and code hundreds of hours of news broadcasts.\footnote{Our
custom software, Metacorps, is freely available at
(REDACTED FOR ANONYMITY).}
% \url{https://github.com/mt-digital/metacorps}.} 
We collected data from the two
most highly rated news shows on each network in each of the two study years,
relying on closed caption data  to create textual transcripts of each show. We
searched each transcript for words that signal violence, namely, \emph{attack},
\emph{beat}, or \emph{hit}. If a violence signal was found, a human reviewer
then manually decided whether it represented metaphorical violence based on
context, annotating the text to identify subject, verb, and object of the phrase
for all uses of metaphorical violence. Annotations were stored along the
transcript, date and time, show, and network to enable later analyses.

We focused primarily on three violence signals which were far the most commonly
used metaphorically among a list of twenty violence words that we initially 
considered. Our initial list was built based on a close reading of newspapers 
and other online news, and cable news transcripts. We assume there is one 
best interpretation of whether or not a statement is metaphorical. For this reason,
we do not calculate inter-rater reliability. We have, however, made our 
full datasets available, including the original phrases found in cable news 
transcripts containing metaphorical and non-metaphorical violence, and all
our annotations (INCLUDE LINK TO DATA). 
Instructions for reviewing our analyses and performing your
own are in the supplement, and on GitHub at 
(REDACTED FOR ANONYMITY).
% \url{https://github.com/mt-digital/metvi-analysis}.

We collected cable television news transcripts indexed by date, network, and
show. We identified and counted daily metaphorical
violence use based on the violence signals \emph{attack}, \emph{hit}, and
\emph{beat} (see Table~\ref{tab:words}). We counted the daily instances of
Democratic presidential candidates (Barack Obama in 2012 and Hillary Clinton in
2016) and Republican presidential candidates (Mitt Romney in 2012 and Donald
Trump in 2016) appearing as the aggressor or victim of metaphorical violence
(see Table~\ref{tab:subjobj}). Sometimes the aggressor and victim of metaphorical 
violence are clear, as a reporter on CNN's \emph{Anderson Cooper 360}
described Clinton criticizing Trump in the first debate as Clinton hitting Trump
\footnote{\url{https://archive.org/details/CNNW_20160928_040000_Anderson_Cooper_360/start/2820/end/2880}}:

\begin{exe}
  \ex Clinton \emph{hit} Trump for voicing support for invading Iraq and calling
    climate change a hoax.
\end{exe}
The subject and object are not always explicitly
specified in a single sentence, but often can be inferred. We include
a reference to the video link on the Internet Archive so the reader can understand
the context which leads us to our inferences.
For instance, a guest on \emph{The Rachel Maddow Show}
described some of Donald Trump's comments as a metaphorical 
\emph{attack} on Hillary Clinton, without saying their names explicitly in the sentence\footnote{\url{https://archive.org/details/MSNBCW_20161021_010000_The_Rachel_Maddow_Show/start/3000/end/3060}}

\begin{exe}
  \ex One joke after another \ldots was a political attack mildly veiled in
  humor.
\end{exe}


\subsection{Dynamical statistical model}

We modeled change in frequency of metaphorical language use as an impulse
function with two states:

\begin{equation} \label{eq:model} 
  f[t] = \begin{cases} f^{(1)} \text{ if } t \in T^{(1)} \\
f^{(2)} \text{ if } t \in T^{(2)} 
  \end{cases} 
\end{equation}
\noindent

Many more complicated models for change in frequency are possible. Here, we
simply used the simplest model of change---that there is one state and then at
some point later there is another.  Of course, in general there many be fewer
than two states or more than two states, and there is no reason to suppose there
would be exactly two.  Nevertheless, we have opted to use the simplest possible
model, supposing there is change.

The dates for which we have data form a time series, $T$, of frequencies of use
for each of the three networks in each of the two election years, six total. All
shows do not air episodes every day. When neither of a network's two shows aired
an episode on a given day, that day is not included in $T$. On a day that is
included in $T$, there may be one or two episodes, so when considering dynamics,
we plot and model the frequency of metaphorical violence use per episode.
Frequency is simply the number of uses in a day divided by the number of
episodes of the shows in that day. The time series are modeled as beginning at a
mean frequency $f^{(1)}$ (State 1), then at some point later, the mean frequency
changes to $f^{(2)}$ (State 2). Model fitting amounts to categorizing dates as
either belonging to State 1 or State 2. These are subsets of $T$, with the State
1's dates denoted $T^{(1)}$ and State 2's dates denoted $T^{(2)}$:

\[ T^{(2)} = \{t \in T: t_{first}^{(2)} \leq t \leq t_{last}^{(2)} \} \]
\noindent and 

\[ T^{(1)} = T~\setminus~T^{(2)}.  \] \noindent To fit parameters
$t_{first}^{(2)}$ and $t_{last}^{(2)}$ to the model to minimize error, we used
Bayesian multi-model inference, which allowed us to quantify the likelihood that
alternate parameterizations would better fit the observed
data~\cite{Burnham2011}. Specifically, to determine the best-fitting model, we
use Bayesian multimodel inference to infer which parameters are most likely to
best represent the system dynamics \cite{Burnham2011}.  Choosing a model with
minimum AIC when all models have the same number of parameters is equivalent to
selecting the model with minimum error, or maximum log-likelihood. Using the AIC
allows us to calculate the relative likelihood of different parameterizations.
Once the minimum AIC is found, call it $\text{AIC}_{min}$, the relative
likelihood that model parameterization $i$ outperforms the model with minimum
AIC is $\text{exp}(\frac{\text{AIC}_{min} - \text{AIC}_i}{2})$. The AIC on
its own tells us nothing about how well the model matches the data, only  how
well the model performs relative to other models. An added feature of using this
inference approach is that it reveals more about the system dynamics than if we
were to simply select and use the  model that minimized error. It also provides
a foundation for future work that considers more complex metaphorical violence
language dynamics. 

Given a model we can calculate the fractional change in frequency, which we
denote by \emph{delta}: 
\begin{equation} 
  \text{\emph{delta}} = \frac{f^{(2)} - f^{(1)}}{f^{(1)}}.  
  \label{eq:delta} 
\end{equation} 
\noindent \emph{Delta},
$t^{(2)}_{first}$, and $t^{(2)}_{last}$ enable us to compare changes in
metaphorical violence language frequency across the networks and over time.
